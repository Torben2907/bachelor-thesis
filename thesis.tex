%%%%%%%%%%%%%%%%%%%%%%%%%%%%%%%%%%%%%%%%%%%%%%%%%%%%%%%%%%%%%%%%%%%%%%%%%%%%%%%%
% Universität Düsseldorf                                                       %
% Lehrstuhl für Softwaretechnik und Programmiersprachen                        %
% Vorlage für Bachelor- und Masterarbeiten                                     %
% Erstellt: 2019-09-03                                                         %
%%%%%%%%%%%%%%%%%%%%%%%%%%%%%%%%%%%%%%%%%%%%%%%%%%%%%%%%%%%%%%%%%%%%%%%%%%%%%%%%
\documentclass{hhuthesis}


%%%%%%%%%%%%%%%%%%%%%%%%%%%%%%%%%%%%%%%%%%%%%%%%%%%%%%%%%%%%%%%%%%%%%%%%%%%%%%%%
%% Einstellungen zur Personalisierung                                         %%
%%                                                                            %%
%% Im Folgenden können Sie Ihre Arbeit personalisieren.                       %%
%%%%%%%%%%%%%%%%%%%%%%%%%%%%%%%%%%%%%%%%%%%%%%%%%%%%%%%%%%%%%%%%%%%%%%%%%%%%%%%%

%% Spracheinstellung
%% Kommentieren Sie die entsprechende Zeile ein bzw. aus.
%% Wir empfehlen jedem sich an einer englischen Arbeit zu versuchen.
% \usepackage[ngerman,english]{babel} % English
\usepackage[english,ngerman]{babel} % Deutsch

%% Ihr Name
\author{Torben Heckes}

%% Der Titel der Arbeit
\title{My Extraordinary Thesis}
% \subtitle{Usually not needed}

%% Der zu erreichende Abschluss, entweder Bachelor oder Master
\graduationtype{Bachelor}
% \graduationtype{Master}

%% Ihr Studienfach
\subject{Informatik}

%% Beginn- und Abgabedaten der Arbeit
\begindate{03.~September~2019} % Beginn
\duedate{03.~Dezember~2019} % Abgabe

%% Erst- und Zweitgutachter
\firstexaminer{Prof.~Dr.~Michael~Leuschel}
\secondexaminer{tbd}

%% Farb- oder Schwarzweißdruck
% Benutzen Sie das Kommando \blackwhiteprint,
% wenn sie in schwarzweiß drucken möchten.
% Im Farbdruck ist jede farbige Seite idR teurer.
% \blackwhiteprint % Kommentarzeichen entfernen für Schwarzweißdruck

%%%%%%%%%%%%%%%%%%%%%%%%%%%%%%%%%%%%%%%%%%%%%%%%%%%%%%%%%%%%%%%%%%%%%%%%%%%%%%%%
%% (Ende) Einstellungen zur Personalisierung                                  %%
%%%%%%%%%%%%%%%%%%%%%%%%%%%%%%%%%%%%%%%%%%%%%%%%%%%%%%%%%%%%%%%%%%%%%%%%%%%%%%%%
%% LaTeX Packages in Nutzung                                                  %%
%%                                                                            %%
%% Im folgenden können Sie für die Niederschrift Ihrer Arbeit benötigte       %%
%% LaTeX-Pakete einbinden.                                                    %%
%% Diese Vorlage kommt bereits mit einigen nützlichen inkludierten Paketen.   %%
%%%%%%%%%%%%%%%%%%%%%%%%%%%%%%%%%%%%%%%%%%%%%%%%%%%%%%%%%%%%%%%%%%%%%%%%%%%%%%%%

%% Macht den \todo-Befehl verfügbar.
%% Hiermit können Sie Abschnitte annotieren,
%% welche weiterer Bearbeitung bedürfen.
\usepackage[textsize=scriptsize]{todonotes}

%% Zeige Zeilennummern in der Arbeit an.
%% Der \linenumbers Befehl muss hierzu aufgerufen werden.
%% Praktisch für Feedback Ihrer potentiellen Korrekturleser!
\usepackage{lineno}
% \linenumbers % <- Kommentar entfernen!


%% Häufig benutzte mathematische Packages.
\usepackage{amsfonts}
\usepackage{amsmath}
\usepackage{amssymb}

\usepackage{siunitx} % \num Befehl zum einfacheren Formatieren von Zahlen.
\usepackage{enumitem} % Leichter konfigurierbare enumerate-Umgebungen.
\usepackage{subcaption} % Unterteilung von Figures in Subfigures.
\usepackage[colorlinks]{hyperref} % Klickbare Links (z.B. Inhaltsverzeichnis).
\usepackage[hypcap=true]{caption} % Setzt Hyperref-Links an den Float-Anfang.
\usepackage{xurl} % \url Kommando für Darstellung von Links
\usepackage{csquotes} % Improved quoting.
\usepackage{microtype} % Verbessertes Kerning zwischen Wörtern.

%% Tabellen
\usepackage{tabularx} % tabularx Umgebung für mehr Kontrolle über Tabellen.
\usepackage{booktabs} % \toprule, \midrule, \bottomrule
\usepackage{multirow}
\usepackage{multicol}
\usepackage{longtable} % Große Tabellen gehen über mehrere Seiten.

%% Quellcode
\usepackage{listings} % Einbindung von Code.

%% Algorithmen in Pseudocode
\usepackage{algorithm} % Float-Umgebung für angegebene Algorithmen.
\usepackage{algorithmicx} % Angabe von Algorithmen in Pseudocode.
\usepackage{algpseudocode} % Standard Pseudocode-Elemente für Algorithmen.

%% Intelligenteres Referenzieren mittels \cref.
%% \languagename um dynamisch zwischen ngerman oder english zu wechseln.
\usepackage[\languagename,capitalize,noabbrev]{cleveref}

%%%%%%%%%%%%%%%%%%%%%%%%%%%%%%%%%%%%%%%%%%%%%%%%%%%%%%%%%%%%%%%%%%%%%%%%%%%%%%%%
%% (Ende) LaTeX Packages in Nutzung                                           %%
%%%%%%%%%%%%%%%%%%%%%%%%%%%%%%%%%%%%%%%%%%%%%%%%%%%%%%%%%%%%%%%%%%%%%%%%%%%%%%%%


\begin{document}
%% Set up title page, declaration of authorship, abstract, acknowledgements
\frontmatter
\makefrontmatter

%%%%%%%%%%%%%%%%%%%%%%%%%%%%%%%%%%%%%%%%%%%%%%%%%%%%%%%%%%%%%%%%%%%%%%%%%%%%%%%%
%% Danksagungen                                                               %%
%%%%%%%%%%%%%%%%%%%%%%%%%%%%%%%%%%%%%%%%%%%%%%%%%%%%%%%%%%%%%%%%%%%%%%%%%%%%%%%%
\begin{acknowledgements}
  Im Falle, dass Sie Ihrer Arbeit eine Danksagung für Ihre Unterstützer
  (Familie, Freunde, Betreuer)
  hinzufügen möchten, können Sie diese hier platzieren.

  Dieser Part ist optional und kann im Quelltext auskommentiert werden.
\end{acknowledgements}
%%%%%%%%%%%%%%%%%%%%%%%%%%%%%%%%%%%%%%%%%%%%%%%%%%%%%%%%%%%%%%%%%%%%%%%%%%%%%%%%
%% (Ende) Danksagungen                                                        %%
%%%%%%%%%%%%%%%%%%%%%%%%%%%%%%%%%%%%%%%%%%%%%%%%%%%%%%%%%%%%%%%%%%%%%%%%%%%%%%%%


\tableofcontents

%% Listings of figures, tables, etc. Delete what is not needed.
\clearpage
\listoftables\thispagestyle{headings}
\listoffigures
\listofalgorithms % Algorithms
\lstlistoflistings % Code Listings

\mainmatter

%%%%%%%%%%%%%%%%%%%%%%%%%%%%%%%%%%%%%%%%%%%%%%%%%%%%%%%%%%%%%%%%%%%%%%%%%%%%%%%%
%% Der Inhalt der Arbeit                                                      %%
%%                                                                            %%
%% Hier können Sie die schriftliche Ausarbeitung ihrer Arbeit                 %%
%% niederschreiben. Der Übersicht halber bietet sich jedoch an, dies in einer %%
%% oder mehreren separaten Dateien zu tun, welche mittels \input eingebunden  %%
%% werden --- wie auch in der Vorlage geschieht.                              %%
%%%%%%%%%%%%%%%%%%%%%%%%%%%%%%%%%%%%%%%%%%%%%%%%%%%%%%%%%%%%%%%%%%%%%%%%%%%%%%%%

%%%%%%%%%%%%%%%%%%%%%%%%%%%%%%%%%%%%%%%%%%%%%%%%%%%%%%%%%%%%%%%%%%%%%%%%%%%%%%%%
% Diese Datei beinhaltet den eigentlichen Inhalt Ihrer Arbeit.
%
% Es bietet sich der Übersicht halber an, die einzelnen Abschnitte jeweils
% in eigene Dateien zu schreiben und mittels \input einzubinden.
% Eine mögliche Verzeichnisstruktur sähe entsprechend so aus:
%
%     thesis/
%     +- tex/
%     |  +- introduction.tex
%     |  +- motivation.tex
%     |  +- experiments.tex
%     |  |  ...
%     |  +- conclusion.tex
%     +- abstract.tex
%     +- contents.tex
%     +- thesis.tex
%%%%%%%%%%%%%%%%%%%%%%%%%%%%%%%%%%%%%%%%%%%%%%%%%%%%%%%%%%%%%%%%%%%%%%%%%%%%%%%%

\section{Einleitung}

Dies ist der Hauptteil Ihrer Arbeit.
Hier leiten Sie grob in das Thema ein, motivieren es und geben einen Ausblick
über das, was Sie in Ihrer Arbeit behandeln werden.

Der Inhalt der Vorlage ist mit Beispielen gefüllt, wie man \LaTeX{}
verwendet und traditionelle Stolpersteine vermeidet.
Lesen Sie die Vorlage gründlich.


\subsection{Makefile}

Im Wurzelverzeichnis finden Sie ein \texttt{Makefile}.
Über das Terminal können Sie die folgenden Befehle aufrufen,
wie in \cref{tab:make} beschrieben.


\begin{table}[h]
  \centering
  \caption{Übersicht der \texttt{Makefile}-Befehle.}%
  \label{tab:make}
  \begin{tabularx}{\textwidth}{lX}
    \toprule
    Befehl & Effekt \\
    \midrule
    \texttt{make} & Kompiliert das PDF und löscht aux-Files. \\
    \texttt{make clean} & Löscht das PDF und dazugehörige aux-Files. \\
    \texttt{make bibtool} & Sortiert \texttt{references.bib}
    und formatiert die Einträge einheitlich. \\
    \texttt{make watch} & Rekompiliert das PDF bei Änderungen und
    hält die Anzeige in Ihrem PDF-Betrachter aktuell. \\
    \bottomrule
  \end{tabularx}
\end{table}


\section{Referenzen und Zitationen}

In der Datei \texttt{references.bib} finden Sie bereits einige Quellen,
die Sie wahrscheinlich zitieren mögen,
wie z.B. die B Methode~\cite{abrial1996b,abrial2010modeling}
oder \textsc{ProB}~\cite{leuschel2003prob,leuschel2008prob}.
Beachten Sie den Artikel ``Common Errors in Bibliographies'' von John Owens.%
\footnote{\url{https://www.ece.ucdavis.edu/~jowens/biberrors.html}}
Zusätzliche, ausführlichere Informationen finden Sie auch in \cref{app:sec:bib}.


\subsection{Referenzen platzieren}

Sie platzieren Referenzen mit \texttt{\textbackslash{}cite{}}.
Diese sollten jeweils hinter dem Namen der zitierten Technik stehen,
oder hinter der zu belegenden Aussage.
Im Falle, dass sich ein gesamter Absatz auf eine einzelne Quelle bezieht,
genügt es die Referenz im ersten Satz anzugeben, solange vom Kontext klar ist,
dass der restliche Absatz sich ebenfalls auf die Quelle bezieht.
Referenzen sind nicht erst am Ende eines gesamten Absatzes zu platzieren.
Ebenso sind sie Teil des Satzes und stehen vor dem abschließenden Punkt.

\begin{itemize}
  \item Gut: \enquote{\textsc{ProB}~\cite{leuschel2003prob} is an
    animator, model checker, and constraint solver for the B method~\cite{abrial1996b}.
    The B method allows to specify, design, and code
    software systems as well as to perform formal proof of their properties.}
  \item Nicht gut: \enquote{\textsc{ProB} is an
    animator, model checker, and constraint solver for the B method~\cite{leuschel2003prob,abrial1996b}.
    The B method allows to specify, design, and code
    software systems as well as to perform formal proof of their properties.}
  \item Nicht gut: \enquote{\textsc{ProB} is an
    animator, model checker, and constraint solver for the B method
    The B method allows to specify, design, and code
    software systems as well as to perform formal proof of their properties.~\cite{leuschel2003prob,abrial1996b}}
\end{itemize}


\subsection{Über Literatur sprechen}

Obwohl die Referenzen innerhalb des Satzes stehen, sind sie keine Wörter.
Sie ersetzen somit nicht die explizite Nennung einer Quelle.
Sollte über ein bestimmtes Papier gesprochen werden, so ist dieses via
Autorennamen zu betiteln. Hierbei gilt:
\begin{itemize}
  \item Es sind nur die Nachnamen zu verwenden,
  \item hinter den Autorennamen ist die Referenz zu setzen,
    falls der Bezug unklar ist,
  \item ab drei oder mehr Autoren wird nur der Erstautor geführt, gefolgt von
    \enquote{et al.}.
\end{itemize}

So schreiben wir
\enquote{SICStus Prolog~\cite{carlsson1988sicstus} wurde von Carlson et al.\
  entwickelt}
oder
\enquote{Leuschel \& Butler~\cite{leuschel2003prob} haben einen Model Checker für die
  B-Methode~\cite{abrial1996b} entwickelt}.
Falsch hingegen ist es, eine Referenz als eigenständiges Wort zu nutzen, wie in
folgendem Negativbeispiel:
\enquote{In \cite{leuschel2003prob} wurde ein Model Checker für die B-Methode entwickelt}.



\section{Bilder und Tabellen}

Bilder und Tabellen sind per se wie von \LaTeX{} bekannt zu setzen.
Wichtig ist, dass sie ausnahmslos im Fließtext referenziert wurden.
Eine nichtreferenzierte Tabelle oder Abbildung kann ebenso ausgelassen werden.
Solche Querverweise werden in \cref{sec:references} besprochen.


\subsection{Bilder}%
\label{sec:figures}

\Cref{fig:initial-draft} ist eine exemplarische Abbildung, auf die an dieser
Stelle im Text verwiesen wird.
Stilistisch ist es meist empfehlenswert innerhalb der
\texttt{figure}-Umgebung ein \texttt{\textbackslash{}centering} zu setzen,
sodass der Inhalt zentriert wird.
In \cref{fig:hhu-logo} wird ein Beispiel der
\texttt{\textbackslash{}subfigure}-Umgebung aus dem
\texttt{\textbackslash{}subcaption}-Paket demonstriert.
Beide Teilabbildungen, \cref{fig:hhu-rgb,fig:hhu-bw},
sind individuell referenzierbar.

\begin{figure}[h]
  \centering
  \includegraphics[width=4cm]{fig/the.png}
  \caption{Initial thesis draft.}%
  \label{fig:initial-draft}
\end{figure}

\begin{figure}[h]
  \begin{subfigure}{.5\textwidth}
    \centering
    \includegraphics[width=4cm]{fig/hhu-logo-rgb.pdf}
    \subcaption{in Farbe}%
    \label{fig:hhu-rgb}
  \end{subfigure}% <- Kommentarzeichen am Ende der Zeile ist wichtig!
  \begin{subfigure}{.5\textwidth}
    \centering
    \includegraphics[width=4cm]{fig/hhu-logo-black.pdf}
    \subcaption{Schwarzweiß}%
    \label{fig:hhu-bw}
  \end{subfigure}% <- Kommentarzeichen am Ende der Zeile ist wichtig!
  \caption{Das neue HHU-Logo.}%
  \label{fig:hhu-logo}
\end{figure}


\subsection{Tabellen}%
\label{sec:tables}

Während bei Abbildungen die \texttt{\textbackslash{}caption}
in aller Regel unter dem Bild steht,
wird sie bei Tabellen typischer Weise darüber platziert, wie bei
\cref{table:truths} zu sehen.

\begin{table}[ht]
  \begin{center}
    \caption{Table of truths.}%
    \label{table:truths}
    \begin{tabular}{lr}
      \toprule
      Fakt                                & Wahrheitsgehalt \\
      \midrule
      booktabs-Tabellen sind hübscher     & 90 \%           \\
      Han Solo schoss zuerst              & 100 \%          \\
      Game of Thrones fand ein gutes Ende & 0 \%            \\
      \bottomrule
    \end{tabular}
  \end{center}
\end{table}

Es ist zu empfehlen, das Paket \texttt{booktabs} zur Formatierung der Tabellen
zu nutzen.
Dieses empfiehlt die Verwendung der Befehle
\texttt{\textbackslash{}toprule},
\texttt{\textbackslash{}midrule} und
\texttt{\textbackslash{}bottomrule},
und rät davon ab, vertikale Linien zu nutzen.
Vergleichen Sie \cref{tab:booktabs-yes,tab:booktabs-yesnt}.

\begin{table}
  \centering
  \caption{Vergleich von \LaTeX{}-Tabellen mit und ohne booktabs.}
  \begin{subtable}{.5\textwidth}
    \centering
    \subcaption{mit booktabs}%
    \label{tab:booktabs-yes}
    \begin{tabular}{lrr}
      \toprule
      Backend & Accuracy & F\textsubscript{1}-Score \\
      \midrule
      CLP(FD) & 0.947 & 0.966 \\
      Z3      & 0.919 & 0.797 \\
      \bottomrule
    \end{tabular}
  \end{subtable}%
  \begin{subtable}{.5\textwidth}
    \centering
    \subcaption{ohne booktabs}%
    \label{tab:booktabs-yesnt}
    \begin{tabular}{|l|r|r|}
      \hline
      Backend & Accuracy & F\textsubscript{1}-Score \\ \hline
      CLP(FD) & 0.947 & 0.966 \\ \hline
      Z3      & 0.919 & 0.797 \\ \hline
    \end{tabular}
  \end{subtable}%
\end{table}



\subsection{Plots}%
\label{sec:plot}

Sie können mithilfe von \texttt{tikz} und \texttt{pgfplots}
Graphen oder Bar Charts erstellen,
wie in \cref{fig:the-plot,fig:long-caption} der gezeigt ist.
Beachten Sie, dass die Vorlage die Cycle Lists
\texttt{hhubwcycle} (schwarzweiß) und \texttt{hhucolorcycle} bereit stellt,
welche die offiziellen HHU-Farben zur Darstellung der verschiedenen Graphen
verwendet.
Diese können in jeglichem \texttt{tikzplot} mittels der Option
\texttt{cycle list name=} gesetzt werden und stehen, je nach Druckeinstellung
innerhalb der PDF via \texttt{\textbackslash{}blackwhiteprint}
entweder automatisch auf \texttt{hhucolorcycle} oder \texttt{hhubwcycle}.

\begin{figure}[ht]
  \centering
  \begin{subfigure}{.5\textwidth}
    \centering
    \begin{tikzpicture}
      \begin{axis}[
        width=\textwidth,
        cycle list name=hhucolorcycle
      ]
        \foreach \y in {0,0.1,...,1} % Wiederholt \addplot mit jeweils anderem \y
          \addplot coordinates {
              ( 1, 3.0 -\y)
              ( 2, 3.25-\y)
              ( 3, 3.5 -\y)
              ( 4, 3.75-\y)
              ( 5, 4.0 -\y)
            };
      \end{axis}
    \end{tikzpicture}
    \subcaption{in den HHU-Farben}
  \end{subfigure}%
  \begin{subfigure}{.5\textwidth}
    \centering
    \begin{tikzpicture}
      \begin{axis}[
        width=\textwidth,
        cycle list name=hhubwcycle % <- Änderung auf schwarzweiß.
      ]
        \foreach \y in {0,0.1,...,1} % Wiederholt \addplot mit jeweils anderem \y
          \addplot coordinates {
              ( 1, 3.0 -\y)
              ( 2, 3.25-\y)
              ( 3, 3.5 -\y)
              ( 4, 3.75-\y)
              ( 5, 4.0 -\y)
            };
      \end{axis}
    \end{tikzpicture}
    \subcaption{in schwarzweiß}
  \end{subfigure}%
  \caption{A beautiful plot.}%
  \label{fig:the-plot}
\end{figure}

\begin{figure}[ht]
  \centering
  \begin{tikzpicture}
    \begin{axis}
      [
        ybar,
        xtick=data,
        enlarge x limits=1,
        symbolic x coords={A, B},
        ymin=0, ymax=100,
        ylabel={$\%$ percentage of bar height},
      ]
      \addplot coordinates {(A,90) (B, 90)};
      \addplot coordinates {(A,75) (B, 75)};
      \addplot coordinates {(A,60) (B, 60)};
      \addplot coordinates {(A,45) (B, 45)};
      \addplot coordinates {(A,30) (B, 30)};
      \legend{blue, orange, green, red, cyan}
    \end{axis}
  \end{tikzpicture}
  \caption[Bar plot with short version of caption for List of Figures]{%
    A really long caption title. This demonstrates how to describe stuff seen
    in the figure, like here, where we see a bar plot showing my favourite
    pies. Nah, actually it shows something completely different.
  }\label{fig:long-caption}
\end{figure}



\section{Querverweise}\label{sec:references}

Für Querverweise, wie zu
\cref{fig:initial-draft,fig:hhu-logo,fig:the-plot,fig:long-caption},
\cref{lst:hello-c,lst:hello-prolog}
oder \cref{alg:minimax}
stehen zwei Möglichkeiten zur Verfügung:
die \LaTeX{}-Standardvariante via \texttt{\textbackslash{}ref},
oder (eleganter) das \texttt{cleveref}-Paket.

Cleveref ermittelt automatisch, welche Art von Element referenziert wird
und fügt den entsprechenden Titel hinzu. Entsprechend sind die in
\cref{tab:cleveref} aufgeführten Beispiele äquivalent.

\begin{table}[ht]
  \centering
  \caption{Vergleich \texttt{\textbackslash{}ref} und \texttt{\textbackslash{}cref}.}%
  \label{tab:cleveref}
  \begin{tabularx}{\textwidth}{XX}
    \toprule
    \LaTeX{} & Darstellung \\
    \midrule
    \lstinline[language=tex]|Abbildung \\ref\{fig:logo\} zeigt das Logo.| &
      Abbildung \ref{fig:hhu-logo} zeigt das Logo. \\
    \lstinline[language=tex]|\\cref\{fig:logo\} zeigt das Logo.| &
      \cref{fig:hhu-logo} zeigt das Logo. \\
    \addlinespace
    \lstinline[language=tex]|Abbildungen \\ref\{fig:plot1\} und|
      \lstinline[language=tex]|\\ref\{plot2\} sind Graphen.| &
    Abbildungen \ref{fig:the-plot} und \ref{fig:long-caption} sind Graphen.\\
    \lstinline[language=tex]|\\Cref\{fig:plot1,plot2\} sind Graphen.| &
    \Cref{fig:the-plot,fig:long-caption} sind Graphen.\\
    \bottomrule
  \end{tabularx}
\end{table}



\section{Formeln}

\Cref{eq:example1} gibt eine referenzierbare Formel an,
während \cref{eq:example2} eine Formel darstellt, die länger ist, als die
Zeile zulässt.

\begin{equation}
  \label{eq:example1}
  2 = 1 + 1
\end{equation}

\begin{multline}
  \label{eq:example2}
  30 = 1 + 1 + 1 + 1 + 1 + 1 + 1 + 1 + 1 + 1 + 1 + 1 + 1 + 1 + 1 + 1 + 1 + 1 \\
      + 1 + 1 + 1 + 1 + 1 + 1 + 1 + 1 + 1 + 1 + 1 + 1
\end{multline}

In der zweizeiligen Gleichung
\begin{equation}
  \label{eq:mlp-stacking}
  \begin{split}
    \hat{y} & = f_2(f_1(x; W); V) \\
            & = f(x; W, V)
  \end{split}
\end{equation}
wurden die Gleichheitszeichen in beiden Zeilen direkt untereinander ausgerichtet
(mittels \texttt{\&} im Quelltext und der \texttt{split}-Umgebung).
Teilen wir \cref{eq:mlp-stacking}, welche den Forward Pass eines Neuronalen
Netzes darstellt,
in mehrere Schritte auf, so erhalten wir (mittels \texttt{align}-Umgebung)
\begin{align}
  a       & = W^\mathsf{T} x \label{eq:fp-act} \\
  h       & = g_1(a) \label{eq:fp-hidden} \\
  o       & = V^\mathsf{T} h \label{eq:fp-out} \\
  \hat{y} & = g_2(o) \label{eq:fp-pred}
  \,\text{,}
\end{align}
wobei \cref{eq:fp-act,eq:fp-hidden,eq:fp-out,eq:fp-pred} jeweils eine eigene
Referenznummer erhalten.


\section{Algorithmen}

Für Algorithmen kann das bereits inkludierte Paket \texttt{algorithmicx}
genutzt werden.
In \cref{alg:minimax} wird exemplarisch eine Implementierung des
Minimax-Algorithmus aufgeführt.
Beachten Sie, dass \cref{line:commented} kommentiert und referenzierbar ist.

\begin{algorithm}
  \caption{Determining the next action by Minimax}%
  \label{alg:minimax}
  \begin{algorithmic}[1]
    \Function{Minimax}{Game State Tree: $G^n$}
      \State bestValue \(\gets -\infty\)
      \State \(\mathit{bestAction} \gets \) NIL
      \ForAll{\(G^n_a \in S(G^n)\)}
        \State \(\mathit{value} = \) \Call{MinimaxValue}{\(G^n_a\), true}
        \If{\(\mathit{value} > \mathit{bestValue}\)}
          \Comment{Aktualisiere besten Wert}\label{line:commented}
          \State \(\mathit{bestValue} \gets \mathit{value}\)
          \State \(\mathit{bestAction} \gets a\)
        \EndIf
      \EndFor
      \State \Return \(\mathit{bestAction}\)
    \EndFunction
    \Statex
    \Function{MinimaxValue}{Game State Tree: $G^n$, Boolean: $\mathit{ourTurn}$}
      \If{\(D(G^n)=0\)}
        \State \Return \Call{Heuristic}{root(\(G^n\))}
      \ElsIf{\(\mathit{ourTurn}\)}
        \State \(\mathit{maxValue} \gets -\infty \)
        \ForAll{\(S \in S(G^n)\)}
          \State \(\mathit{newValue} \gets \) \Call{MinimaxValue}{\(S\), false}
          \State \(\mathit{maxValue} \gets
            \max(\mathit{newValue}, \mathit{maxValue})\)
        \EndFor
        \State \Return \(maxValue\)
      \Else
        \State \(minValue \gets +\infty \)
        \ForAll{\(S \in S(G^n)\)}
          \State \(\mathit{newValue} \gets \) \Call{MinimaxValue}{\(S\), true}
          \State \(\mathit{minValue} \gets
            \min(\mathit{newValue}, \mathit{minValue})\)
        \EndFor
        \State \Return \(minValue\)
      \EndIf
    \EndFunction
  \end{algorithmic}
\end{algorithm}


\section{Source Code Listings}

\Cref{lst:hello-c,lst:hello-python} zeigen ein `Hello World'-Programm,
je in C und Python.
\Cref{lst:hello-prolog} zeigt ein Prolog-Prädikat, welches eine Liste in zwei
Teile teilen kann.

\begin{lstlisting}[
  float, caption={Hello World in C.}, label={lst:hello-c}, language=C
]
#include <stdio.h>

int main(int argc, char[] *args){
  printf("Hello World!\n");
  // And done!
}
\end{lstlisting}

\begin{lstlisting}[
  float, caption={Totally minimal Hello World in Python.},
  label={lst:hello-python}, language=Python
]
def hello_world():
  print("Hello World"!)

if __name__ == "__main__":
  hello_world()
\end{lstlisting}

\begin{lstlisting}[
  float, caption={Prolog implementation of \texttt{split/4}},
  label={lst:hello-prolog}, language=Prolog
]
% Split list into two parts (length of first list given).
%
% ?- split([a,b,c,d,e,f,g,h,i,k], 3, L1, L2).
% L1 = [a,b,c]
% L2 = [d,e,f,g,h,i,k]
%
split(L, N, L1, L2) :-
  length(L1, N),
  append(L1, L2, L).
\end{lstlisting}


\section{Todonotes}

Es bietet sich an, während der Verschriftlichung Gebrauch von dem
\texttt{todonotes}-Package zu machen.\todo[]{Lernen, wie man mit todonotes umgeht.}

Abgesehen davon, dass es erlaubt die PDF mit offenen Todos zu annotieren,
sind sie ein guter Weg um potentziellen Korrekturlesern zu kommunizieren,
welche Teile ohnehin noch nicht ausgearbeitet sind.
Des Weiteren lässt sich mit
\texttt{\textbackslash{}listoftodos} eine Übersicht der noch offenen Todos
im Dokument anzeigen

\listoftodos

\todo[inline]{Lerne was man machen kann, wenn man einzeln stehende Todos braucht.}
\todo[inline]{Lerne ebenfalls, \texttt{\textbackslash missingfigure} zu nutzen.}
\todo[inline]{Verschaffung eines Überblicks der in der Vorlage inkludierten Pakete.}


\subsection{Missingfigure}

Mittels \texttt{\textbackslash{}missingfigure} lassen sich bereits Abbildungen
im PDF darstellen, die noch erstellt werden müssen. Dies ermöglicht bereits einen
ersten Eindruck, wie das Layout um die Abbildung herum aussehen wird,
wie \cref{fig:missing} exemplarisch zeigt.

\begin{figure}
  \centering
  \missingfigure{Plot is still to be done, but the results from the HPC are not
    yet available.}
  \caption{Finale Laufzeiten meiner unfassbar guten Auswertung.}%
  \label{fig:missing}
\end{figure}


\section{Häufige Fehler}

Der wohl häufigste Fehler, den neue \LaTeX-Nutzer machen, ist eine falsche
Verwendung von Anführungszeichen.
Um dem Vorzubeugen sowie konsistente Anführungszeichen zu nutzen,
empfehlen wir den Einsatz von \texttt{\textbackslash{}enquote}
aus dem bereits inkludierten \texttt{csquotes}-Pakt.
% Dieses erlaubt ebenfalls eine gezielte und automatische Unterscheidung
% zwischen den \foreignquote{english}{englischen} und
% \foreignquote{german}{deutschen} Anführungszeichen.
Wird hingegen, wie in der Programmierung üblich, \textquotedbl verwendet,
versucht \LaTeX{} daraus einen Umlaut zu erzeugen.
So wird beispielsweise \texttt{\textquotedbl{}a} zu ä.

\Cref{tab:quotes} gibt eine Übersicht über verschiedene Möglichkeiten
Anführungszeichen zu setzen und demonstriert die falsche Verwendung von
\textquotedbl.

\begin{table}[ht]
  \centering
  \caption{Demonstration von csquotes.}%
  \label{tab:quotes}
  \begin{tabularx}{\textwidth}{XX}
    \toprule
    \LaTeX{} & Darstellung \\
    \midrule
    \texttt{Ein \textquotedbl{}Wort\textquotedbl{} in \textquotedbl{}Anführungszeichen\textquotedbl{}} &
      Ein "Wort" in "Anführungszeichen" \\ \addlinespace
    \texttt{Ein \textbackslash{}glqq Wort
        in \textbackslash{}glq Anführungszeichen\textbackslash{}grq\textbackslash{}grqq}&
      Ein \glqq Wort in \glq Anführungszeichen\grq\grqq \\ \addlinespace
    \texttt{Ein \textbackslash{}enquote\{Wort
        in \textbackslash{}enquote\{Anführungszeichen\}}\} &
      Ein \enquote{Wort in \enquote{Anführungszeichen}} \\
    \bottomrule
  \end{tabularx}
\end{table}

\section{Conclusion}

Am Ende der Arbeit werden noch einmal die erreichten Ergebnisse
zusammengefasst und diskutiert.


%% Dieser Part kann auskommentiert werden, sollte kein Anhang nötig sein.
%% Der \appendix-Befehl leitet hierbei den Anhang ein.
 \appendix
 \section{Zusätzliche Informationen}

Hier können Sie Ihren Anhang definieren.

Achten Sie darauf, dass der Anhang in Ihrer \texttt{thesis.tex}
initial auskommentiert ist.
Der entsprechende Part befindet sich nahe dem Ende der Datei.
Entfernen Sie bei Bedarf die Kommentierung um den Anhang nutzen zu können.


\section{Nutzung}

Der Anhang wird wie die Abschnitte des Hauptteils der Arbeit gestaltet,
also mit \texttt{\textbackslash section} Befehlen.

  \subsection{Unterabschnitte}
  Die Verwendung von Unterabschnitten im Anhang
  mittels \texttt{\textbackslash subsection}
  funktioniert ebenfalls!

\section{Bibliographie}%
\label{app:sec:bib}

Während ihrer Literatur-Recherche werden Sie vermutlich hauptsächlich auf vier Arten
an Quellen stoßen, die Sie in der \verb|.bib|-Datei unterbringen:
Konferenz-Artikel, Sammlungen an Kapiteln von mehreren Autoren, Journal-Artikel und Monographien.
Manche Bibtex-Einträge (etwa von Google Scholar) enthalten nicht alle Informationen
bzw. nicht im selben Format.

\textbf{Wichtig: Eine saubere Bibliographie besteht aus vollständigen und einheitlichen Einträgen!}

Im Folgenden wird beschrieben, was Sie hier für eine möglichst saubere Literaturliste beachten sollten.

\subsection{Konferenzartikel}

In der Informatik wird die meiste Forschung auf Konferenzen vorgestellt.
Hier reichen Autoren ihre Arbeiten ein, andere Forscher begutachten diese
und anhand dessen werden die besten Arbeiten für Konferenzvorträge ausgewählt.
In der Regel gibt es dazu Tagungsbände, die sogenannten \emph{Proceedings},
in denen die vorgestellten Forschungsartikel gesammelt werden.

\begin{figure}
    \centering
    \includegraphics[scale=.5]{fig/prob-springer.png}
    \caption[Screenshot von \url{https://link.springer.com}]{
      Screenshot von \url{https://link.springer.com/chapter/10.1007/978-3-540-45236-2_46}}%
    \label{fig:prob-springer}
\end{figure}

Ein Beispiel ist der Artikel von Leuschel und Butler zu \textsc{ProB}~\cite{leuschel2003prob}.
In \cref{fig:prob-springer} sind die wichtigen Informationen für den Bibtex-Eintrag zu finden,
die Sie auf jeden Fall angeben sollten.
Ein Beispiel für einen passenden Bibtex-Eintrag lautet wie folgt (hier wird der \verb|InProceedings|-Typ verwendet):

\begin{verbatim}
@InProceedings(leuschel2003prob,
  Author	= {Leuschel, Michael and Butler, Michael},
  Title		= {{ProB}: A Model Checker for {B}},
  Year		= 2003,
  Month		= sep,
  Booktitle	= {{FME} 2003: Formal Methods},
  Series	= {LNCS},
  Volume	= 2805,
  Pages		= {855--874},
  Publisher	= {Springer},
  Address	= {Berlin, Heidelberg}
)
\end{verbatim}

\paragraph{Title, Author, Pages.} Diese Einträge sollten selbsterklärend sein.
\paragraph{Booktitle.} Hier gibt es mehrere valide Möglichkeiten.
Auf dem Buch selbst steht \enquote{FME 2003: Formal Methods}.
Akzeptabel wären auch \enquote{Proceedings FME 2003},
\enquote{Proceedings Formal Methods Europe (2003)},
\enquote{International Symposium of Formal Methods Europe. Pisa Italy, September 8-14, 2003, Proceedings}
oder Mischformen.
Wichtig ist, dass erkenntlich ist, zu welcher Konferenz aus welchem Jahr der Tagungsband stammt.
\paragraph{Year.} Leider erscheinen die Proceedings nicht unbedingt im selben Jahr wie die Konferenz selbst.
Daher ist auch das Jahr der Veröffentlichung anzugeben.
\paragraph{Series und Volume}. Proceedings vom Springer-Verlag erscheinen in der Regel in einer Reihe.
Üblich sind die LNCS (Lecture Notes in Computer Science). Darin erhalten sie auch eine Nummer, um sie eindeutig zu identifizieren.
Es gibt aber auch andere Reihen (z.B. LNAI oder CCIS); manchmal werden Tagungsbände auch mehreren Reihen zugeordnet.
Andere Herausgeber haben keine solche Reihe --- dann fallen diese Einträge weg.
\paragraph{Publisher.} Der Herausgeber der Proceedings.
Die meisten Tagungsbände werden vom Springer-Verlag oder der ACM veröffentlicht.
Hier gibt es auch mehrere valide Angaben (z.B. \enquote{Springer} bzw. \enquote{Springer-Verlag} oder \enquote{ACM} bzw. \enquote{Association for Computing Machinery}).
Diese sollten in Ihrer Bibliographie einheitlich sein.



\subsection{Sammelbände}

Es gibt einige Sammlungen an Artikeln, die als thematisches Buch veröffentlicht werden,
allerdings nicht aus einer Konferenz entstehen.
Ein Beispiel ist die Festschrift zu Egon Börgers 75.\ Geburtstag.
Darin findet man unter anderem den folgenden Artikel:

\begin{verbatim}
@InCollection(Leuschel2021,
  Author    = {Leuschel, Michael},
  Title     = {Spot the Difference: A Detailed Comparison Between {B}
                and Event-{B}},
  Booktitle = {Logic, Computation and Rigorous Methods},
  Publisher = {Springer},
  Year      = 2021,
  Volume    = 12750,
  Series    = {Lecture Notes in Computer Science},
  Pages     = {147--172}
)
\end{verbatim}

Insgesamt ist dies sehr ähnlich zum Konferenzartikel; deshalb wird her nicht näher auf die einzelnen Schlüssel eingegangen.
Auch dieses Buch wurde in der LNCS-Reihe veröffentlicht.

\subsection{Journal-Artikel}

Artikel in Journals werden in der Regel als hochwertiger als Konferenz-Artikel angesehen
und sind daher als Quelle zu bevorzugen.
Hier ist die Zeit für die Gutachten in der Regel länger und es wird genauer hingesehen.
Häufig entstehen Journal-Artikel aus Konferenzartikeln und sind ausführlichere Versionen.
Manchmal wird aber auch ohne eine Konferenz-Version direkt in einem Journal eingereicht.

Für die Bibliographie werden viele ähnliche Einträge wie beim Konferenzartikel verwendet,
allerdings ist der Typ hier \verb|Article|.

\begin{verbatim}
@Article(leuschel2008prob,
  Author	= {Leuschel, Michael and Butler, Michael},
  Title		= {{ProB}: An Automated Analysis Toolset for the {B} Method},
  Year		= 2008,
  Month		= mar,
  Journal	= {International Journal on Software Tools for
             Technology Transfer},
  Volume	= 10,
  Pages		= {185--203},
  Number	= 2
)
\end{verbatim}

\paragraph{Title, Author, Pages, Year.} Diese Einträge sollten wieder selbsterklärend sein.
\paragraph{Journal.} Die Artikel werden in einem Journal veröffentlicht, das einen Namen hat.
Hier gibt es in der Regel auch etablierte Abkürzungen (wie etwa \enquote{STTT} für
\enquote{International Journal on Software Tools for Technology Transfer}).
Das Format sollte hier auch einheitlich sein.
\paragraph{Volume und Number.} Journal-Artikel werden meist gesammelt periodisch veröffentlicht.
In der Regel wird die Volume-Zahl pro Jahr erhöht;
innerhalb einer Volume gibt es dann häufig mehrere Veröffentlichungen, die mit der \enquote{issue number} hochgezählt werden.
Wird das Journal also vierteljährlich veröffentlicht, geht der Number-Eintrag bis 4 hoch.
Manchmal gibt es auch \enquote{special issues} zu einem bestimmten Thema oder als Sammlung an Artikeln,
die aus einer bestimmten Konferenz hervorgingen.

\subsection{Monographien}

Einige Bücher werden von vollständig von wenigen Autoren geschrieben.
Hier werden insgesamt recht wenig Informationen benötigt,
wie etwa in dem Beispiel hier:

\begin{verbatim}
@Book(abrial1996b,
  Author	= {Abrial, Jean-Raymond},
  Title		= {The {B}-Book: Assigning Programs to Meanings},
  Year		= 1996,
  Publisher	= {Cambridge University Press},
  Address	= {New York, NY, USA}
)
\end{verbatim}

Leicht andere Arten der Monographien sind Abschlussarbeiten.
Hier haben Master- und Doktorarbeiten eigene Typen mit selbsterklärenden Schlüsseln.
Falls Sie eine Bachelorarbeit zitieren möchten, können Sie auch den Typen \verb|MastersThesis| verwenden.

\begin{verbatim}
@MastersThesis(eulynx_ma,
  Author    = {Abdul Rasheeq},
  Title     = {An Approach To Improve {SysML} Railway Specification
                Using {UML-B} And Event-{B}},
  School    = {Frankfurt University of Applied Sciences},
  Year      = 2019
)
\end{verbatim}

\begin{verbatim}
@PhDThesis(nummenmaa2013executable,
  Author    = {Nummenmaa, Timo},
  Title     = {Executable formal specifications in game development:
                Design, validation and evolution},
  School    = {University of Tampere},
  Year      = 2013
)
\end{verbatim}


%%%%%%%%%%%%%%%%%%%%%%%%%%%%%%%%%%%%%%%%%%%%%%%%%%%%%%%%%%%%%%%%%%%%%%%%%%%%%%%%
%% (Ende) Der Inhalt der Arbeit                                               %%
%%%%%%%%%%%%%%%%%%%%%%%%%%%%%%%%%%%%%%%%%%%%%%%%%%%%%%%%%%%%%%%%%%%%%%%%%%%%%%%%


\backmatter

\clearpage
\bibliography{references}
%% Depending on Language, use german alphadin or original alpha
\iflanguage{ngerman}{
  \bibliographystyle{alphadin}
}{
  \bibliographystyle{alpha}
}

\end{document}
